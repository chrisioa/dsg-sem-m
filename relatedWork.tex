\section{Related Work}
The Cloud-native Network function is a young field of research that continues the trend from breaking up rigidity in large scale networks. Consequently, few works exist capturing the essence of the new Cloud-native paradigm in the network domain.

Imadali \textit{et al.} have given a short insight into the design principles that need to be taken into account when defining Cloud-native Network Functions. Following that, they provide insight into existing NFV projects and discuss a new cloud service model, namely \textit{5G as a Service} that could for instance be used in a scenario where Radio Access Networks are being deployed and shared among multiple tenants. Following that, they defined a \textit{5G CN-VNF} framework, used to make networking infrastructure accessible. By no means is this production ready, but provides a good introduction to the subject and shows how tools to ease development in this regards can be designed \cite{cn5gvnf}.

A whitepaper by the 5G-PPP titled \textit{From Webscale to Telco, the Cloud Native Journey} \cite{5gppp} emphasizes the relevance of this new computing paradigm especially for the telecommunications industry. The structure of the work shows what the main enablers are for implementing and realize the new use and business cases of fifth generation networks. Software architecture principles as defined in the twelve factor app and the microservice architecture are among them, as well as open source software following useful standards.

Finally, Telco and networking providers also postulate their views on CNFs especially in the context of 5G. Among the companies that have authored such a paper are Cisco \cite{CNF}, Huawei \cite{evolutionnfv}, Ericsson \cite{ericsson} and many more. Financial interests often lie behind this kind of publication from enterprises, which can be a double-edged sword: On the one hand, the motivation might be to promote services and devices to drive sales. On the other hand, this provides an interesting insight into which technologies are deemed possibly realizable and profitable.