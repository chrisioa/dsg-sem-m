\section{Conclusion}

\subsection{Summary}
In this paper, the basic concepts of how functions are being developed, deployed and managed in large scale networks have been explored. An overview and evaluation of concepts that have been put forward to tackle the limitations of traditional networks by applying the cloud native principles to the network domain has been given.
Since this domain is very specific to the internet service and telecommunications providers, special requirements play a crucial role like reliability and high performance. The reliability aspect has contributed to the fact that there is a certain hesitation to switch from proven, old concepts to more modern and flexible solutions. This can be exemplified with packaging functionalities inside physical boxes, to virtually abstracting from general purpose hardware and bundling the functionality in software artifacts. NFV was proposed in 2012 and the concept of using virtual machines, organized via Openstack has since been implemented in many networks. The recent advancements of the public (and private) cloud providers in rendering their offers progressively cheaper and remote service models have caused a paradigm shift in software architecture and engineering. Using advancements like microservices, design considerations like from the 12-factor app and CI/CD pipelines and softwarized network technologies like SDN to further automation, lays the foundation for more intricate business cases for fifth generation networks.

\subsection{Future Work}
%Future Work typically means what YOU would do if you had infinite time for your research. Not what the technology will be in the future. Try to use a different name for that section. Or did I read that wrongly?
The field of Cloud-native Network Functions is just starting to get to the point where first, interesting results can be observed. Often these are limited to proofs-of-concept and calls to action, but other projects have started implementing NFV-Infrastructures which have not yet been exhaustively researched. Additionally, the technical limitations of Kubernetes and Docker in the domain of the telco domain have not been touched upon in this paper, since it serves as an introduction to this complex subject. An interesting field of investigation would be the specific interconnection of the containers and pods and how application traffic can be guaranteed to flow with high performance. As an outlook, DPDK (Data Plane Development Kit) an open sourced software project developed by Intel has found wide usage in accelerating application packet processing. 