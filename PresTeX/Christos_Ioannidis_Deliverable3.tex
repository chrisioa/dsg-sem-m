%===============================================================================
% Zweck:    KTR-Präsentation-Vorlage
% Erstellt: 15.04.2013
% Update:   04.07.2016
% Autor:    M.G.
%===============================================================================
\RequirePackage[hyphens]{url}
\newcommand\ratio{169}
\documentclass[10pt,aspectratio=\ratio,
%draft,
%handout,
compress
]{beamer}
\usepackage{verbatim}
\usepackage{pifont}
\usepackage{graphicx}
\usepackage{tikz}
\usepackage{mdframed}
\usepackage{listings}

\newcommand\meta{./meta}
\input{\meta/config/commands}


\def\signed #1{{\leavevmode\unskip\nobreak\hfil\penalty50\hskip2em
  \hbox{}\nobreak\hfil(#1)%
  \parfillskip=0pt \finalhyphendemerits=0 \endgraf}}

\newsavebox\mybox
\newenvironment{aquote}[1]
  {\savebox\mybox{#1}\begin{fancyquotes}}
  {\signed{\usebox\mybox}\end{fancyquotes}}


\input{\meta/config/hyphenation}

\setbeamertemplate{caption}[numbered]
%\numberwithin{figure}{section}

\begin{document}
  %===============================================================================
  % Zum Kompilieren latexmk ausführen.
  %	Konfiguration in texmaker: Options -> Configure Texmaker -> Quick Build -> Select Latexmk + ViewPD
  %	Entsprechende Informationen in den config/metainfo verändern
  % Zur Auswahl der Sprache im folgenden Befehl
  % ngerman für deutsch eintragen, english für Englisch.
  %===============================================================================
\selectlanguage{english}
\ifnum\ratio<169
\frame{\titlepage}
\else
\frame[plain]{\titlepage}
\fi

%\AtBeginSection[]
%{
%  \frame<handout:0>
%  {
%    \frametitle{Outline}
%    \tableofcontents[currentsection,hideallsubsections]
%  }
%}

\AtBeginSubsection[]
{
  \frame<handout:0>
  {
    \frametitle{Outline}
    \tableofcontents[sectionstyle=show/hide,subsectionstyle=show/shaded/hide,subsubsectionstyle=hide]
  }
}

\AtBeginSubsubsection[]
{
  \frame<handout:0>
  {
    \frametitle{Outline}
    \tableofcontents[sectionstyle=show/hide,subsectionstyle=show/shaded/hide,subsubsectionstyle=show/shaded/hide]
  }
}

\newcommand<>{\highlighton}[1]{%
  \alt#2{\structure{#1}}{{#1}}
}

\newcommand{\icon}[1]{\pgfimage[height=1em]{#1}}

\section*{}
\phantomsection
\begin{frame}{Content}
\tableofcontents
\end{frame}
%%%%%%%%%%%%%%%%%%%%%%%%%%%%%%%%%%%%%%%%%
%%%%%%%%%% Content starts here %%%%%%%%%%
%%%%%%%%%%%%%%%%%%%%%%%%%%%%%%%%%%%%%%%%%

\section{Network Functions  -  4G Architecture}

%\frame{\frametitle{Introduction - 4G Architecture}
%	\begin{figure}
%		\includegraphics[width=0.5\linewidth]{images/raninterfaces.png}
%		\caption{Radio Access Network interfaces \cite{dahlman20164g}}
%	\end{figure}
%}

\frame{\frametitle{Network Functions - 4G Architecture}
	\begin{figure}
		\includegraphics[width=0.25\linewidth]{images/epc.png}
		\caption{Evolved Packet Core architecture \cite{dahlman20164g}, Home Subscriber Service, Mobility Management Entity, Serving Gateway, Packet Data Network Gateway}
	\end{figure}
}



\section{Physical Network Service Composition}
\frame{\frametitle{Physical Network Service Composition} \framesubtitle{Numbers}
	\begin{figure}
		\includegraphics[width=0.8\linewidth]{images/middleboxesNumbers.png}
		\caption{Number of physical middleboxes in enterprise and public networks  \cite{sherry2016middleboxes}}
	\end{figure}
}

\frame{\frametitle{Physical Network Service Composition} \framesubtitle{Cost}
	\begin{figure}
		\includegraphics[width=0.8\linewidth]{images/middleboxesCost.png}
		\caption{Cost of physical middleboxes  \cite{sherry2016middleboxes}}
	\end{figure}
}

\section{Virtualization}
\subsection{Motivation}
\frame{\frametitle{Network Virtualization} \framesubtitle{Motivation}
	Disadvantages of Physical Network Service Composition
	\begin{itemize}
		\item Physical appliances need physical processing (acquisition, deployment, setup...)
		\item Manual, vendor specific configuration
		\item Long time to market
		\item Inflexible service composition
		\item hard and costly maintenance
	\end{itemize}
}

\frame{\frametitle{Network Virtualization} \framesubtitle{Motivation}
	 \begin{figure}
	 	\includegraphics[width=0.6\linewidth]{images/nfv.png}
	 	\caption{Traditional vs virtual network functions\cite{nfv_wp}}
	 \end{figure}
}


\subsection{NFV}

\frame{\frametitle{Network Function Virtualization} \framesubtitle{NFVI Architecture}
	\begin{figure}
		\includegraphics[width=0.7\textwidth]{images/nfvi.png}
		\caption{Network Function Virtualization Infrastructure \cite{ li2015software}}
	\end{figure}
}

\frame{\frametitle{Network Function Virtualization} \framesubtitle{Reference Architecture}
	\begin{figure}
		\includegraphics[width=0.6\textwidth]{images/nfv_ref_arch.png}
		\caption{The ETSI NFV Reference Architecture \cite{etsi1etsi}}
	\end{figure}
}

\frame{\frametitle{Network Function Virtualization} \framesubtitle{End-to-end services}
	\begin{figure}
		\includegraphics[width=0.7\textwidth]{images/nfv_forwarding.png}
		\caption{NFV forwarding to establish end-to-end connections \cite{ etsi1etsi}}
	\end{figure}
}


\subsection{CNF}
\frame{\frametitle{Cloud-native Network Function Virtualization} \framesubtitle{Evolution}
	\begin{figure}
		\includegraphics[width=1\textwidth]{images/vnctocnf.png}
		\caption{Evolution from VNF to CNF architecture \footnote{\url{https://community.infoblox.com/t5/Cloud-Native/En-Route-to-Cloud-Native-Network-Functions/ba-p/16311}}}
	\end{figure}
}

\frame{\frametitle{CNF Example Architecture}
	\begin{figure}
		\includegraphics[width=0.9\textwidth]{images/nfv4arm3.png}
		\caption{Example architecture using Kubernetes to provide an NFV Infrastructure on ARM devices \cite{nfv4arm}}
	\end{figure}
}


\section{Summary \& Future Work}
\frame{\frametitle{Summary \& Future Work}
	Summary
	\begin{itemize}
		\item Large-scale networks have long used middleboxes
		\item Breaking up vertical integration to enable flexibility and reduce cost
		\item NFV concept and architectures have been introduced
		\item SDN for programmable networks
		\item CNF to enable unprecedented automation
	\end{itemize}
	
	Future Work
	\begin{itemize}
		\item Explore technical details of networking of Docker and K8 (dpdk, ovs, etc)
		\item Investigate real life adaption (Orange indicates $\sim$ 50~\% virtualization)
		\item Explore realization of DevOps for CNF
	\end{itemize}
}


%%%%%%%%%%%%%%%%%%%%%%%%%%%%%%%%%%%%%%%%%
%%%%%%%%%% References          %%%%%%%%%%
%%%%%%%%%%%%%%%%%%%%%%%%%%%%%%%%%%%%%%%%%


%% Last frame
\frame{
  \vspace{2cm}
  {\huge Thanks for your attention!}

  \vspace{20mm}
  \nocite*

  \begin{flushright}
    Christos Ioannidis

    \structure{\footnotesize{\href{mailto:christos.ioannidis@stud.uni-bamberg.de}{christos.ioannidis@stud.uni-bamberg.de}}}
  \end{flushright}
}


\begin{frame}[allowframebreaks]{References - Selection}
	\def\newblock{\hskip .11em plus .33em minus .07em}
	\scriptsize
	\bibliographystyle{IEEEtran}
	\bibliography{literature/literature.bib}
	\normalsize
\end{frame}

\frame{\frametitle{Software Defined Networking}	
	\begin{figure}
		\includegraphics[width=0.8\textwidth]{images/sdn.png}
		\caption{Separation of the data/control planes \cite{li2015software}}
	\end{figure}
}

\frame{\frametitle{Software Defined Networking}	
	\begin{figure}
		\includegraphics[width=0.8\textwidth]{images/sdnAPIs.png}
		\caption{SDN controller APIs \cite{jarschel2014interfaces}}
	\end{figure}
}

\end{document}
